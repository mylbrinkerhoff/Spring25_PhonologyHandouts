% !TEX TS-program = lualatex
% !TEX encoding = UTF-8 Unicode

\documentclass[12pt, letterpaper]{article}

%%BIBLIOGRAPHY- This uses biber/biblatex to generate bibliographies according to the
%%Unified Style Sheet for Linguistics
\usepackage[main=american, german]{babel}% Recommended
\usepackage{csquotes}% Recommended
\usepackage[backend=biber,
        style=unified,
        maxcitenames=3,
        maxbibnames=99,
        natbib,
        url=false]{biblatex}
\addbibresource{.bib}
\setcounter{biburlnumpenalty}{100}  % allow URL breaks at numbers
% \setcounter{biburlucpenalty}{100}   % allow URL breaks at uppercase letters
% \setcounter{biburllcpenalty}{100}   % allow URL breaks at lowercase letters

%%TYPOLOGY
\usepackage[svgnames]{xcolor} % Specify colors by their 'svgnames', for a full list of all colors available see here: http://www.latextemplates.com/svgnames-colors
%\usepackage[compact]{titlesec}
%\titleformat{\section}[runin]{\normalfont\bfseries}{\thesection.}{.5em}{}[.]
%\titleformat{\subsection}[runin]{\normalfont\scshape}{\thesubsection}{.5em}{}[.]
\usepackage[hmargin=1in,vmargin=1in]{geometry}  %Margins          
\usepackage{graphicx}	%Inserting graphics, pictures, images 		
\usepackage{stackengine} %Package to allow text above or below other text, Also helpful for HG weights 
\usepackage{fontspec} %Selection of fonts must be ran in XeLaTeX or LuaLaTeX
\usepackage{amssymb} %Math symbols
\usepackage{amsmath} % Mathematical enhancements for LaTeX
\usepackage{setspace} %Linespacing
\usepackage{multicol} %Multicolumn text
\usepackage{enumitem} %Allows for continuous numbering of lists over examples, etc.
\usepackage{multirow} %Useful for combining cells in tables 
\usepackage{booktabs}
\usepackage{hanging}
\usepackage{fancyhdr} %Allows for the 
\pagestyle{fancy}
\fancyhead[L]{\textit{Phonology I Handout}} 
\fancyhead[R]{\textit{\today}} 
\fancyfoot[L,R]{} 
\fancyfoot[C]{\thepage} 
\renewcommand{\headrulewidth}{0.4pt}
\setlength{\headheight}{14.5pt} % ...at least 14.49998pt
% \usepackage{fourier} % This allows for the use of certain wingdings like bombs, frowns, etc.
% \usepackage{fourier-orns} %More useful symbols like bombs and jolly-roger, mostly for OT
\usepackage[colorlinks,allcolors={black},urlcolor={blue}]{hyperref} %allows for hyperlinks and pdf bookmarks
% \usepackage{url} %allows for urls
% \def\UrlBreaks{\do\/\do-} %allows for urls to be broken up
\usepackage[normalem]{ulem} %strike out text. Handy for syntax
\usepackage{tcolorbox}
\usepackage{datetime2}
\usepackage{caption}
\usepackage{subcaption}

%%FONTS
% \setmainfont{Libertinus Serif}
\setmainfont{Linux Libertine O}
% \setsansfont{Libertinus Sans}
\setsansfont{Linux Biolinum O}
% \setmonofont{Libertinus Mono}
\setmonofont[Scale=MatchLowercase]{Libertinus Mono}

%%PACKAGES FOR LINGUISTICS
%\usepackage{OTtablx} %Generating tableaux with using TIPA
% \usepackage[noipa]{OTtablx} % Use this one generating tableaux without using TIPA
%\usepackage[notipa]{ot-tableau} % Another tableau drawing packing use for posters.
% \usepackage{linguex} % Linguistic examples
% \usepackage{langsci-linguex} % Linguistic examples
\usepackage{langsci-gb4e} % Language Science Press' modification of gb4e
% \usepackage{langsci-avm} % Language Science Press' AVM package
\usepackage{tikz} % Drawing Hasse diagrams
% \usepackage{pst-asr} % Drawing autosegmental features
% \usepackage{pstricks} % required for pst-asr, OTtablx, pst-jtree.
% \usepackage{pst-jtree} 	% Syntax tree draawing software
% \usepackage{tikz-qtree}	% Another syntax tree drawing software. Uses bracket notation.
% \usepackage[linguistics]{forest}	% Another syntax tree drawing software. Uses bracket notation.
% \usepackage{ling-macros} % Various linguistic macros. Does not work with linguex.
% \usepackage{covington} % Another linguistic examples package.
\usepackage{leipzig} %	Offers support for Leipzig Glossing Rules

%%LEIPZIG GLOSSING FOR ZAPOTEC
\newleipzig{el}{el}{elder}	% Elder pronouns
\newleipzig{hu}{hu}{human}	% Human pronouns
\newleipzig{an}{an}{animate}	% Animate pronouns
\newleipzig{in}{in}{inanimate}	% Inanimate pronouns
\newleipzig{pot}{pot}{potential}	% Potential Aspect
\newleipzig{cont}{cont}{continuative}	% Continuative Aspect
% \newleipzig{pot}{pot}{potential}	% Potential Aspect
\newleipzig{stat}{stat}{stative}	% Potential Aspect
\newleipzig{and}{and}{andative}	% Andative Aspect
\newleipzig{ven}{ven}{venative}	% Venative Aspect
% \newleipzig{res}{res}{restitutive}	% Restitutive Aspect
\newleipzig{rep}{rep}{repetitive}	% Repetitive Aspect

%%TITLE INFORMATION
\title{TITLE}
\author{Mykel Loren Brinkerhoff}
\date{\today}

%%MACROS
\newcommand{\sub}[1]{\textsubscript{#1}}
\newcommand{\supr}[1]{\textsuperscript{#1}}
\providecommand{\lsptoprule}{\midrule\toprule}
\providecommand{\lspbottomrule}{\bottomrule\midrule}
\newcommand{\fittable}[1]{\resizebox{\textwidth}{!}{#1}}

\makeatletter
\renewcommand{\paragraph}{%
  \@startsection{paragraph}{4}%
  {\z@}{0ex \@plus 1ex \@minus .2ex}{-1em}%
  {\normalfont\normalsize\bfseries}%
}
\makeatother
\parindent=10pt


\begin{document}

%%If using linguex, need the following commands to get correct LSA style spacing
%% these have to be after  \begin{document}
    % \setlength{\Extopsep}{6pt}
    % \setlength{\Exlabelsep}{9pt}		%effect of 0.4in indent from left text edge
%%

%% Line spacing setting. Comment out the line spacing you do not need. Comment out all if you want single spacing.
%	\doublespacing
%	\onehalfspacing

\begin{center}
     {\Large \textbf{Week 5: Natural Classes and Rules}}
\end{center}
%\maketitle
%\maketitleinst
\thispagestyle{fancy}

% \tableofcontents

%------------------------------------
\section{Santiago Laxopa Zapotec} \label{}
%------------------------------------

Santiago Laxopa Zapotec (SLZ; \textit{Dille'xhunh Laxup} [di˨ʒe͡ʔ˨ʐun˨ lːa˨ʂːupʰ˦]) is a variety of Zapotec spoken by about 1000 people in the town of Santiago Laxopa, Oaxaca, Mexico. It is a member of the Zapotecan language family, which is part of the larger Oto-Manguean language family. 

\begin{table}[!h]
	\centering
	\caption{Consonant inventory for Santiago Laxopa Zapotec}
	\label{tab:SLZcons}
	\fittable{
	\begin{tabular}{llcccccccc}
	\lsptoprule
		 & bilabial & alveolar  & post- & retroflex & palatal &velar &labio-  &  uvular \\
		 &&&alveolar&  &&&velar& \\
	\midrule
	stop        & pː b & tː d & & & & kː g & kʷː gʷ & \\
	fricative   &      & sː z & ʃː ʒ & ʂː ʐ & çː & & & ʁ \\
	affricate 	&       & t͡sː d͡z & t͡ʃː &  & & & & \\
    nasal		&	mː  & nː n & & & & & & \\
	lateral  	&       & lː l & & & & & & \\
	trill		&       & rː & & &  & &  & \\ 			
	approximate &       & & & & & & w & \\ 
	\lspbottomrule
	\end{tabular}
	}
\end{table}

SLZ has a four phonemic vowels and three phonation types, which are shown in Table~\ref{tab:SLZ_vowel_chart}.\footnote{The high back vowel /u/ is realized as [o] in some contexts, but this is not phonemic and is restricted to certain lexical items. Additionally, older speakers tend to realize /u/ as [o] in all contexts, while younger speakers have a more stable realization of /u/ as [u] save for a few lexical items like \textit{me'edo'} [mːḛdo͡ʔ] `baby'.

Additionally, there is a fourth phonemic phonation type: checked. The checked vowels are realized as a complex segment consisting of a vowel followed by a glottal stop, (e.g., /a͡ʔ/). These will not be discussed in this handout.} 

\begin{table}[h!]
    \centering
    \caption{Vowel qualities in Santiago Laxopa Zapotec. }
    \label{tab:SLZ_vowel_chart}
    \begin{tabular}{lccc}
        \lsptoprule
        &  front& central  & back \\
        \midrule 
        high   	&  i i̤ ḭ  &     &   u ṳ ṵ \\
        mid    	&  e e̤ ḛ &   	& 	\\
        low   	&     &  a a̤ a̰ 	&	  \\
        \lspbottomrule
    \end{tabular}
\end{table}

SLZ also has a number of phonemic tones that are realized as differences in pitch. There are a total of five tones: High (˦), Mid (˧), Low (˨), Rising (˧˦), Falling (˦˨), which won't be discussed in this handout. 

%------------------------------------
\section{Natural Classes} 
%------------------------------------

Some natural classes (for this language) are listed below. For each natural class, Provide the features that define the class. Remember to use the least number of features possible.

\begin{enumerate}
	\item {}[pː, tː, kː, kʷː, sː, ʃː, ʂː, çː, t͡sː, t͡ʃː]
	\item {}[mː, nː, lː, rː]
	\item {}[b, d, g, gʷ, z, ʒ, ʐ, d͡z, ʁ, n, l]
	\item {}[i, e, i̤, e̤, ḭ, ḛ]
	\item {}[z, ʒ, ʐ]
\end{enumerate}

%------------------------------------
\section{Listing Segments} 
%------------------------------------

List all the segments that belong to the following natural classes.

\begin{enumerate}
	\item {}[+labial]
	\item {}[+long, +sonorant]
	\item {}[+continuant, -long]
	\item {}[+high, -constructed glottis, -spread glottis]
	\item {}[-high, -low, -constructed glottis, +spread glottis]
	\item {}[+dorsal, +round, -syllabic]
	\item {}[+dorsal, +back, -high, +continuant, -syllabic]
	\item {}[+coronal, -long, -continuant, +delayed release]
\end{enumerate}

%------------------------------------
\section{Rules} 
%------------------------------------

Write rules to capture the following generalizations. Use as few features as possible to pick out all and only the relevant inputs, outputs, and environments.

\begin{enumerate}
	\item Long obstruents become short and aspirated word finally (e.g., /tːapː/ → [tːapʰ] `four'; /çːetː/ → [çːetʰ] `tortilla') 
	\vspace{0.75cm}
	\item Voiced obstruents become devoiced and fricatives word finally (e.g., /ʂːa̰g/ → [ʂːa̰x] `topil'; /pːad͡ziuʐ/ → [pːad͡ziuʂ] `hello/goodbye')
	\vspace{0.75cm}
	\item Long sonorants become voiceless when they are word initial and followed by a consonant (e.g., /rːmːed͡zʁw/ → [r̥ːmːed͡zʁw̥] `medicine'; /lːnːi/ → [l̥ːnːi] `party')
	\vspace{0.75cm}
	\item Obstruents agree in voicing with the following obstruent (e.g., /btːiʂḛ/ → [ptːiʂːḛ] `trogon'; /ʂː-ʒin=a͡ʔ/ → [ʐːʒina͡ʔ] `my work')
\end{enumerate}

%------------------------------------
\section{Morphological Decomposition} 
%------------------------------------

Isthmus Zapotec ([dɪ˨d͡ʒa˨ˈzaˑ˨]), is an Oto-Manguean language spoken in Oaxaca, Mexico by 104,000 speakers on the Isthmus of Tehuantepec.

\ea
\begin{multicols}{3} 
	\ea {}[palu] `stick'
	\ex {}[kuːba] `dough'
	\ex {}[tapa]	`four'
	\ex {}[geta] `tortilla'
	\ex {}[bere] `chicken'
	\ex {}[doʔo] `rope'
	\ex {}[spalube] `his stick'
	\ex {}[skubabe] `his dough'
	\ex {}[stapabe] `his four'
	\ex {}[sketabe] `his tortilla'
	\ex {}[sperebe] `his chicken'
	\ex {}[stoʔobe] `his rope'
	\ex {}[spalulu] `your stick'
	\ex {}[skubalu] `your dough'
	\ex {}[stapalulu] `your four'
	\ex {}[sketalulu] `your tortilla'
	\ex {}[sperele] `your chicken'
	\ex {}[stoʔole] `your rope'
	\z 
\end{multicols}
\z 

Using the data above, answer the following questions:

\begin{enumerate}
	\item Isolate the morphemes in the data above for the following morphemes:
	\begin{enumerate}
		\item Possessive marker: 
		\item Third-person singular:
		\item Second-person plural:
	\end{enumerate}
	\item What are the allomorphs for the following morphemes?
	\begin{enumerate}
		\item stick: 
		\item dough:
		\item four:
		\item tortilla:
		\item chicken:
		\item rope:
	\end{enumerate}

	\item What phonological rule accounts for the alternations in the data above? Write the rule in the form of a phonological rule using features.
\end{enumerate}
%------------------------------------
%BIBLIOGRAPHY
%------------------------------------

%\singlespacing
%\nocite{*}
\printbibliography[heading=bibintoc]

\end{document}