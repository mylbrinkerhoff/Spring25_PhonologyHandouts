% !TEX TS-program = lualatex
% !TEX encoding = UTF-8 Unicode

\documentclass[12pt, letterpaper]{article}

%%BIBLIOGRAPHY- This uses biber/biblatex to generate bibliographies according to the
%%Unified Style Sheet for Linguistics
\usepackage[main=american, german]{babel}% Recommended
\usepackage{csquotes}% Recommended
%\usepackage[backend=biber,
%        style=unified,
%        maxcitenames=3,
%        maxbibnames=99,
%        natbib,
%        url=false]{biblatex}
%\addbibresource{.bib}
%\setcounter{biburlnumpenalty}{100}  % allow URL breaks at numbers
% \setcounter{biburlucpenalty}{100}   % allow URL breaks at uppercase letters
% \setcounter{biburllcpenalty}{100}   % allow URL breaks at lowercase letters

%%TYPOLOGY
\usepackage[svgnames]{xcolor} % Specify colors by their 'svgnames', for a full list of all colors available see here: http://www.latextemplates.com/svgnames-colors
%\usepackage[compact]{titlesec}
%\titleformat{\section}[runin]{\normal ont\bfseries}{\thesection.}{.5em}{}[.]
%\titleformat{\subsection}[runin]{\normalfont\scshape}{\thesubsection}{.5em}{}[.]
\usepackage[hmargin=1in,vmargin=1in]{geometry}  %Margins          
\usepackage{graphicx}	%Inserting graphics, pictures, images 		
\usepackage{stackengine} %Package to allow text above or below other text, Also helpful for HG weights 
\usepackage{fontspec} %Selection of fonts must be ran in XeLaTeX or LuaLateX
\usepackage{amssymb} %Math symbols
\usepackage{amsmath} % Mathematical enhancements for LaTeX
\usepackage{setspace} %Linespacing
\usepackage{multicol} %Multicolumn text
\usepackage{enumitem} %Allows for continuous numbering of lists over examples, etc.
\usepackage{multirow} %Useful for combining cells in tablesbrew 
\usepackage{hanging}
\usepackage{fancyhdr} %Allows for the 
\pagestyle{fancy}
\fancyhead[L]{\textit{Phonology I Section Handout}} 
\fancyhead[R]{\textit{\today}} 
\fancyfoot[L,R]{} 
\fancyfoot[C]{\thepage} 
\renewcommand{\headrulewidth}{0.4pt}
\setlength{\headheight}{14.5pt} % ...at least 14.49998pt
% \usepackage{fourier} % This allows for the use of certain wingdings like bombs, frowns, etc.
% \usepackage{fourier-orns} %More useful symbols like bombs and jolly-roger, mostly for OT
\usepackage[colorlinks,allcolors={black},urlcolor={blue}]{hyperref} %allows for hyperlinks and pdf bookmarks
% \usepackage{url} %allows for urls
% \def\UrlBreaks{\do\/\do-} %allows for urls to be broken up
\usepackage[normalem]{ulem} %strike out text. Handy for syntax
\usepackage{tcolorbox}
%\usepackage{todonotes} % Creates todo marginalia

%%FONTS
\setmainfont{Libertinus Serif}
\setsansfont{Libertinus Sans}
\setmonofont[Scale=MatchLowercase]{Libertinus Mono}

%%PACKAGES FOR LINGUISTICS
%\usepackage{OTtablx} %Generating tableaux with using TIPA
%\usepackage[noipa]{OTtablx} % Use this one generating tableaux without using TIPA
%\usepackage[notipa]{ot-tableau} % Another tableau drawing packing use for posters.
\usepackage{linguex} % Linguistic examples
% \usepackage{langsci-linguex} % Linguistic examples
%\usepackage{langsci-gb4e} % Language Science Press' modification of gb4e
% \usepackage{langsci-avm} % Language Science Press' AVM package
\usepackage{tikz} % Drawing Hasse diagrams
% \usepackage{pst-asr} % Drawing autosegmental features
\usepackage{pstricks} % required for pst-asr, OTtablx, pst-jtree.
% \usepackage{pst-jtree} 	% Syntax tree draawing software
% \usepackage{tikz-qtree}	% Another syntax tree drawing software. Uses bracket notation.
\usepackage[linguistics]{forest}	% Another syntax tree drawing software. Uses bracket notation.
% \usepackage{ling-macros} % Various linguistic macros. Does not work with linguex.
% \usepackage{covington} % Another linguistic examples package.
\usepackage{leipzig} %	Offers support for Leipzig Glossing Rules

%%LEIPZIG GLOSSING FOR ZAPOTEC
\newleipzig{el}{el}{elder} % Elder pronouns
\newleipzig{hu}{hu}{human} % Human pronouns
\newleipzig{an}{an}{animate} % Animate pronouns
\newleipzig{in}{in}{inanimate} % Inanimate pronouns
\newleipzig{pot}{pot}{potential} % Potential Aspect
\newleipzig{cont}{cont}{continuative} % Continuative Aspect
\newleipzig{stat}{stat}{stative} % Stative Aspect
\newleipzig{and}{and}{andative} % Andative Aspect
\newleipzig{ven}{ven}{venative} % Venative Aspect
% \newleipzig{res}{res}{restitutive} % Restitutive Aspect
\newleipzig{rep}{rep}{repetitive} % Repetitive Aspect

%%TITLE INFORMATION
\title{TITLE}
\author{Mykel Loren Brinkerhoff}
\date{\today}

%%MACROS
\newcommand{\sub}[1]{\textsubscript{#1}}
\newcommand{\supr}[1]{\textsuperscript{#1}}
\newcommand{\tab}{\hspace{1cm}}

\makeatletter
\renewcommand{\paragraph}{%
  \@startsection{paragraph}{4}%
  {\z@}{0ex \@plus 1ex \@minus .2ex}{-1em}%
  {\normalfont\normalsize\bfseries}%
}
\makeatother
\parindent=10pt


\begin{document}

%%If using linguex, need the following commands to get correct LSA style spacing
%% these have to be after  \begin{document}
    % \setlength{\Extopsep}{6pt}
    % \setlength{\Exlabelsep}{9pt}		%effect of 0.4in indent from left text edge
%%

%% Line spacing setting. Comment out the line spacing you do not need. Comment out all if you want single spacing.
%	\doublespacing
%	\onehalfspacing

\begin{center}
     {\Large \textbf{Week 3: Phonemic Analysis}}
     \vspace{6pt}
\end{center}
% \maketitle
%\maketitleinst
\thispagestyle{fancy}

% \tableofcontents
%------------------------------------
\section*{Goals} \label{sec:Goals}
%------------------------------------

\begin{itemize}
    \item Practice with phonemic and distributional analysis
    \item Learn how to write rules in a phonological analysis
    \item Learn how to determine the underlying representation of a phoneme
\end{itemize}

%------------------------------------
\section*{Persian} \label{sec:persian}
%------------------------------------

Persian (also known as Farsi) is a member of the Iranian branch of the Indo-European language family, with around 110 million speakers mainly concentrated in Iran, Afghanistan, and Tajikistan.

\ex. 
\begin{multicols}{3}
    \a. {}[ærteʃ] `army' 
    \b. {}[ahaɾi] `starched'
    \b. {}[ahar̥] `starch'
    \b. {}[farsi] `Persian'
    \b. {}[ʃiɾini] `pastry'
    \b. {}[axær̥] `last'
    \b. {}[qædri] `a little bit'
    \b. {}[beɾid] `go'
    \b. {}[tʃedʒur̥] `better'
    \b. {}[rah] `road'
    \b. {}[biɾæŋ] `pale'
    \b. {}[ʃir̥] `city'
    \b. {}[rast] `right'
    \b. {}[boɾos] `hairbrush'
    \b. {}[ræŋ] `paint'
    \b. {}[tʃeɾa] `why'
    \b. {}[riʃ] `beard'
    \b. {}[daɾid] `you have'
    \b. {}[ruz] `day'
    \z. 
\end{multicols}
    
\begin{itemize}
    \item Look at the distribution of [r], [ɾ], and [r]. Write out the environments in which they appear.
    \vspace{1.75cm}
    \item Are these sounds in complementary or contrastive distribution?
    \vspace{1.75cm}
    \item Are they allophones of the same or different phonemes?
    \vspace{1.75cm}
    \item If they are allophones of the same phoneme, write a rule to derive the non-underlying allophones in the specific environment(s) where they occur.
\end{itemize}

\newpage

%------------------------------------
\section*{Kaqchikel} \label{sec:kaqchikel}
%------------------------------------

Kaqchikel is a Mayan language, with over 450,000 native speakers across the western highlands of Guatemala. The following data is representative of the San Lucas Tolimán dialect of Kaqchikel, spoken on the shores of Lake Atitlán (about two hours from the capital, Guatemala City). Focus on the distribution of [b’ w f]. The symbol [b’]indicates a voiceless bilabial implosive. Some details of pronunciation have been suppressed for readability.

\ex. 
\begin{multicols}{2}
    \a. {}[b’ukut] `shoe'
    \b. {}[tef] `cold'
    \b. {}[b’iʃ] `song'
    \b. {}[xob’] `rain'
    \b. {}[aweʃ] `your teeth'
    \b. {}[ulef] `land'
    \b. {}[ak'wal] `young person'
    \b. {}[utif] `coyote'
    \b. {}[qab'ij] `our name'
    \b. {}[tʃuf] `smelly'
    \b. {}[axaf] `god'
    \b. {}[ʃajab'] `sandal'
    \b. {}[waqiʔ] `six'
    \b. {}[kab'] `sugar'
    \b. {}[joxwɨr] `we are sleeping'
    \b. {}[wiʔaj] `hair'
    \b. {}[kof] `hard'
    \b. {}[wuquʔ] `seven'
    \b. {}[b'aq] `bone'
    \b. {}[xkowir] `it hardened'
    \b. {}[jab'e] `you are going'
    \b. {}[ʃiʃb'e] `you (pl) went'
    \z. 
\end{multicols}

\begin{itemize}
    \item Look at the distribution of [b’], [w], and [f]. Write out the environments in which they appear and give a descriptive generalization about their distribution.
    \vspace{2cm}
    \item How many underlying phonemes do [b’], [w], and [f]belong to? Justify your answer, using specific examples from the data given above, and the terminology of phonemic analysis.
    \vspace{2cm}
    \item If any of these three sounds belong to a shared underlying phoneme, which allophone represents the `basic’ variant? Why?
    \vspace{2cm}
    \item Write any phonological rules needed for your analysis to generate the correct distribution of surface allophones.
\end{itemize}
\newpage

%------------------------------------
\section*{Gascon} \label{sec:gascon}
%------------------------------------

Gascon is a \textit{lenga d'òc} language spoken in southwestern France. It is closely related to Occitan, Catalan, and Provençal. For this dataset, focus on the distribution of [b d ɡ β ð ɣ] 

\ex. 
\begin{multicols}{2}
    \a. [brẽn] `endanger'
    \b. [bako] `cow'
    \b. [ũmbro] `shadow'
    \b. [krãmbo] `room'
    \b. [dilys] `Monday'
    \b. [dũŋko] `until'
    \b. [duso] `sweet'
    \b. [taldepãn] `leftover bread'
    \b. [pũnde] `to lay eggs'
    \b. [ɡuteʒa] `flow'
    \b. [ẽŋɡwãn] `this year'
    \b. [puðe] `to be able'
    \b. [ɡat] `cat'
    \b. [lũŋɡ] `long'
    \b. [saliβo] `saliva'
    \b. [noβi] `husband'
    \b. [aβe] `to have'
    \b. [ʃiβaw] `horse'
    \b. [byðɛt] `gut'
    \b. [eʃaðo] `hoe'
    \b. [biɣar] `mosquito'
    \b. [riɣut] `he laughed'
    \b. [aɡro] `sour'
    \b. [ʒuɣet] `he played'
    \z. 
\end{multicols}

\begin{itemize}
    \item Look at the distribution of [b d ɡ β ð ɣ]. Write out the environments in which they appear and give a descriptive generalization about their distribution. 
    \vspace{4cm}
    \item Write a single phonological rule that can account for your analysis to generate the correct distribution of surface allophones.
    \vspace{2cm}
    \item Show a derivation for `sour', `he laughed', `to be able', and `to lay eggs'
\end{itemize}

%------------------------------------
%\section{} \label{}
%------------------------------------


%------------------------------------
%BIBLIOGRAPHY
%------------------------------------

%\singlespacing
%\nocite{*}
%\printbibliography[heading=bibintoc]

\end{document}